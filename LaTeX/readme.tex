\documentclass[11pt]{article}
\usepackage{amssymb}
\usepackage{amsmath}
\usepackage[margin=1in]{geometry}
\begin{document}

\begin{center}
	CMPS 109 -- Homework 5 -- Hex with Monte Carlo AI
\end{center}

\paragraph{Compilation Instructions: } \hspace{0pt} \\
	I have included a \texttt{makefile} for you. Just calling \texttt{make} will 
	compile with your machines default version of \texttt{g++}. \texttt{make all}
	will recompile everything. Calling \texttt{make c11}
	will compile everything using the \texttt{-std=c++11} flag. I included a
	threading option which I will explain later. It requires a \texttt{c++11} compliant compiler. I have
	tested these compilation options on my personal computer and on the unix server.
	(Note: the c++11 and threading does not work on the unix server so do not try that on unix) \\
	Compiler: \texttt{g++ (MacPorts gcc48 4.8.0\_0) 4.8.0} \\
	Hardware: \texttt{2.4 GHz i5 2Core}

\paragraph{Input: } \hspace{0pt} \\
	There is no command line parameters. You can change the board size in \texttt{main()}
	and the number of simulations in \texttt{hex.cpp} at the top of the file. \\
	Each user turn will take 2 values, an $(x,y)$ coordinate.
	
\paragraph{Monte Carlo AI Description: } \hspace{0pt} \\
	 My implementation of the AI uses a Monte Carlo Method. For every un-played spot,
	  make a copy and make a move on that board. Now I have a copy 
	 of the original board with one more move for every un-played spot on the original board.
	 Now for each of those I randomly generate the rest of the board and test who wins. I do this
	 random generation the amount of times specified in \texttt{hex.cpp} as \texttt{SIMSPERMOVE}.
	 All the way, I am keeping track of the amount of wins for the board with the move. \\\\
	 The theory being, a certain move on a certain board will give the computer an innate advantage 
	 in the game. This advantage will show itself in the random generation.
	 
\paragraph{My implementation of threads} \hspace{0pt} \\
	Essentially what I have implemented, is that each board
	on which we are doing simulations (one for each un-played space) will have its own thread.\\\\
	
	\noindent As I noted earlier this requires some special compilation instructions. First of all you 
	need a \texttt{c++11} compliant compiler. Personally I used \texttt{g++ 4.8} on my machine.
	Furthermore, I put in some preprocessor directives so that we can choose wether to 
	implement threading or not. To enable threading, uncomment out \texttt{\#define C11} at the
	top of \texttt{hex.cpp}. Then compile with the \texttt{-std=c++11} flag on \texttt{g++} by 
	running \texttt{make c11}. 
	
	 
	 
\end{document}